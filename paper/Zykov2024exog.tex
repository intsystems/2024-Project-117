\documentclass[12pt, twoside]{article}
\usepackage{jmlda}
\newcommand{\hdir}{.}
\usepackage{natbib}

\begin{document}

\title
    [Поиск зависимостей в биомеханических системах] % краткое название; не нужно, если полное название влезает в~колонтитул
    {Поиск зависимостей в биомеханических системах}
\author
    [Зыков Т.А.] % список авторов (не более трех) для колонтитула; не нужен, если основной список влезает в колонтитул
    {Зыков~Т.А.} % основной список авторов, выводимый в оглавление
    [Зыков~Т.А.$^1$, Дорин~Д.Д.$^2$, Стрижов~В.В.$^{3}$] % список авторов, выводимый в заголовок; не нужен, если он не отличается от основного
\email
    {zykov.ta@phystech.edu}
\organization
    {$^1$Chair of Data analysis; $^2$, $^3$Intelligent systems;}
\abstract
    {% Данный текст является шаблоном статьи, подаваемой для публикации в журнале <<Машинное обучение и анализ данных>>.

    
   Исследуется проблема восстановления зависимости между показаниями датчиков фМРТ и восприятием внешнего мира человеком. Проводится анализ зависимости между последовательностью снимков фМРТ и звуковым рядом. Требуется предложить метод прогнозирования показаний фМРТ по прослушиваемому звуковому ряду. При прогнозировании сложноорганизованных временных рядов, зависящих от экзогенных факторов и имеющих множественную периодичность, связи между рядами устанавливаются с помощью метода сходящегося перекрестного отображения и тестом Гренджера. 
}
\maketitle

\section{Введение}

Функциональная магнитно-резонансная томография (фМРТ) \citep{puras2014neurovisualization} — метод нейровизуализации, который измеряет активность мозга путем выявления изменений, связанных с кровотоком. Метод нейровизуализации фМРТ находит применение для анализа корреляции активности мозга с такими заболеваниями как аутизм и болезнь Альцгеймера, а также для прогнозирования таких заболеваний как черепно-мозговая травма и имеет потенциал для терапии \citep{fmriapplications}. Он дает представление о том, какие области мозга участвуют в конкретных психических процессах или задачах, измеряя уровень BOLD (Blood-oxygen-level-dependent). BOLD это показатель, используемый в функциональной магнитно-резонансной томографии, который отражает изменения кровотока и уровня насыщения кислородом в мозге \citep{bold}. 

Работа посвящена восстановлению зависимости между снимками фМРТ
и звуковым рядом. Предполагается, что такая зависимость существует. Кроме того, предполагается, что между снимком и звуковым рядом есть постоянная задержка во
времени\citep{menon1999spatial}. Время задержки выступает в качестве гиперпараметра модели. Предлагается метод аппроксимации показаний фМРТ по прослушиваемому звуковому ряду. Проводится первичный анализ влияния типа звукового ряда (музыка или простой диалог) на изменения уровня BOLD и анализ зависимости снимков фМРТ от звукового ряда.

В данной работе рассмотрим одно из ограничений BOLD\citep{menon1999spatial}. Временное разрешение -- это наименьший период времени нейронной активности который с высокой точностью можно определить с помощью фМРТ. Время измерений фМРТ происходит с задержкой, что затрудняет регистрацию быстрых нейронных событий. Кроме того, при выборе модели важно учитывать структуру временных рядов: у некоторых вокселей прослеживается тренд на протяжении всей звуковой дорожки или неоднородность шума, при которых можно использовать оценки Уайта\citep{hetero}.

Для извлечения признаков из звуковой дорожки можно использовать разные подходы\citep{audiofeaturetrends}. В этой работе рассматривается мел-кепстральные коэффициенты (MFCC). Это характеристики аудиосигнала которые получаются из спектрограммы и используются для анализа звука и обработки речи. Преимуществом такого представления является небольшая размерность.

В качестве датасета будем использовать данные\citep{Berezutskaya2022}, собранные у большой группы людей во время просмотра короткометражного аудиовизуального фильма. Он включает записи фМРТ 30 участников в возрасте от 7 до 47 лет. Данные были получены с использованием богатого аудиовизуального стимула. Звуковая дорожка состоит из диалогов персонажей и музыкальных вставок.


\section{Постановка задачи}
Требуется предложить метод прогнозирования показаний фМРТ по прослушиваемому звуковому ряду. 
Обозначим частоту снимков фМРТ $\mu \in \mathbb{R}$. Задана последовательность снимков
\begin{equation}
	\label{eq1}
	\bS = [\bs_1, \ldots, \bs_{\mu t}], \quad\
	\bs_{\ell} \in \mathbb{R}^{X \times Y \times Z},
\end{equation}
где $X, Y$ и $Z$~--- размерности воксельного изображения.

Задана частота дискретизации $\nu \in \mathbb{R}$, количество каналов $k \in \mathbb{N}$ и продолжительность $t \in \mathbb{R}$ аудиоряда.
Задан непрерывный по времени сигнал
\begin{equation}
	\label{eq2}
	\bP = [\bp_1, \ldots, \bp_{\nu t}], \quad\
	\bp_{\ell} = 
        \begin{pmatrix}
        p_\ell^1\\
        p_\ell^2 \\
        \vdots \\
        p_\ell^k\\
        \end{pmatrix}, \quad\
	p_{\ell}^k \in \mathbb{R},
\end{equation}

Задача состоит в построении отображения, которое бы учитывало задержку $\Delta t$ между
снимком фМРТ и аудиорядом, а также предыдущие томографические показания. Формально, необходимо
найти такое отображение $\mathbf{f}$, что
\begin{equation}
	\label{eq3}
	\mathbf{f}(\bp_1, \ldots, \bp_{k_{\ell} - \nu \Delta t}; \bs_1, \ldots, \bs_{\ell-1}) = \bs_{\ell},
	\ \ell = 1, \ldots, \mu t,
\end{equation}
где для $\ell$-го снимка фМРТ номер соответствующего сигнала $k_{\ell}$ определяется по формуле
\begin{equation}
	\label{eq4}
	k_{\ell} = t\nu = \dfrac{\ell}{\mu}\nu.
\end{equation}

Эмбеддингами аудиоряда будут мел-кепстральные коэффициенты \citep{mfcc}. То есть
для каждого экземпляра сигнала имеем вектор размерности $\mathbf{d}$:
\begin{equation}
	\label{eq5}
 \bx_{\ell} = [x^{\ell}_1, \ldots, x^{\ell}_{d}]\T \in \mathbb{R}^{d}, \ {\ell} = 1, \ldots, \dfrac{\nu t}{h}. 
\end{equation}
Будем восстаналивать функцию $\mathbf{f}$, в предположении марковского свойства. 

\begin{equation}
	\label{eq6}
	\mathbf{f}(\bx_{k_\ell - 
 \nu\Delta t - g}, \ldots, \bx_{k_\ell - 
 \nu\Delta t}) = \bs_\ell - \bs_{\ell-1} = \bdelta_\ell
	\ \ell = 2, \ldots, \mu t,
\end{equation}
где $\bdelta_{\ell} = [s^{\ell}_{ijk} - s^{\ell-1}_{ijk}] = [\delta^{\ell}_{ijk}] \in \mathbb{R}^{X \times Y \times Z}$~--- разность между двумя последовательными снимками.


Учитывая \eqref{eq4}, суммарное число пар (сигнал, снимок)
равно $N = \mu (t - \Delta t)$. Таким образом, для каждого вокселя задана выборка
\[ \mathfrak{D}_{ijk} = \{(\bx_{\ell}, \delta^{\ell}_{ijk}) \ | \ {\ell} = 2, \ldots, N \}. \]

Поставлена элементарная задача восстановления регрессии
\begin{equation}
	\label{eq7}
	y_{ijk}: \mathbb{R}^{d} \to \mathbb{R}.
\end{equation}

Рассмотрим каждый воксель независимо $Y_{ijk} \in \mathbb{R}^{N}$ - воксели, $X \in \mathbb{R}^{N \times d}$. Предполагаемая зависимость
\begin{equation}
	\label{eq8}
	Y_{ijk} = X\theta + \varepsilon,
\end{equation}
где $\theta \in \mathbb{R}^d$ - коэффициенты модели, $\varepsilon \sim N(0, \Sigma)$ - шум.

Требуется найти параметры $\widehat\theta$, доставляющие максимум функции правдоподобия при заданных гиперпараметрах $\Delta t$ и $d$, где $d$ -- размерность MFCC:
\begin{equation}
	\label{eq9}
	L_X(\theta) = \prod\limits_{v=1}^N p_\theta(Y_{ijk}^v) \longrightarrow \mylim{max}_\theta
\end{equation}

\section{Предварительные сведения}



%%%% если имеется doi цитируемого источника, необходимо его указать, см. пример в \bibitem{article}
%%%% DOI публикации, зарегистрированной в системе Crossref, можно получить по адресу http://www.crossref.org/guestquery/
\bibliographystyle{unsrt}
\bibliography{references.bib}

\end{document}
