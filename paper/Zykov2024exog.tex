\documentclass[12pt, twoside]{article}
\usepackage{jmlda}
\newcommand{\hdir}{.}

\begin{document}

\title
    [Поиск зависимостей в биомеханических системах] % краткое название; не нужно, если полное название влезает в~колонтитул
    {Поиск зависимостей в биомеханических системах}
\author
    [Зыков Т.А.] % список авторов (не более трех) для колонтитула; не нужен, если основной список влезает в колонтитул
    {Зыков~Т.А.} % основной список авторов, выводимый в оглавление
    [Зыков~Т.А.$^1$, Дорин~Д.Д.$^2$, Тихонов~Д.М.$^{3}$] % список авторов, выводимый в заголовок; не нужен, если он не отличается от основного
\email
    {zykov.ta@phystech.edu}
\organization
    {$^1$Chair of Data analysis; $^2$, $^3$Intelligent systems;}
\abstract
    {% Данный текст является шаблоном статьи, подаваемой для публикации в журнале <<Машинное обучение и анализ данных>>.

    
   Исследуется проблема восстановления зависимости между показаниями датчиков фМРТ и восприятием внешнего мира человеком. Проводится анализ зависимости между последовательностью снимков фМРТ и звуковым рядом. Требуется предложить метод прогнозирования показаний фМРТ по прослушиваемому звуковому ряду и улучшить качество предсказания с помощью видеоряда. При прогнозировании сложноорганизованных временных рядов, зависящих от экзогенных факторов и имеющих множественную периодичность, связи между рядами устанавливаются с помощью метода сходящегося перекрестного отображения и тестом Гренджера. 
    
 %    Аннотация описывает основную цель работы,
 %    особенности предлагаемого подхода и~основные результаты.
 %    Сведения, содержащиеся в заглавии статьи, не должны повторяться в тексте авторского резюме.
 %    В аннотации не должно быть ссылок на литературу и, по возможности, формул.
	
	% Также необходимо представить расширенную структурированную аннотацию на английском языке объемом 200--300 слов.	
	% Английская аннотация может не быть дословным переводом русского текста и должна быть написана хорошим английским языком.
	
	% В титульном заголовке необходимо указать полный, официально принятый, переводной вариант названия организации.
	% Указывать нужно только ту часть названия, которая относится к понятию юридического лица,
	% не вписывая названий кафедры, лаборатории или другого структурного подразделения внутри организации.
	% Необходимо указать полный юридический адрес, или, как минимум, город и страну.
 	
 % 	При выборе ключевых слов основным критерием является их потенциальная ценность для выражения содержания документа или для его поиска.
	% В качестве ключевых слов могут использоваться термины из названия, аннотации, вступительной и заключительной части текста статьи.
 % 	При подборе ключевых слов рекомендуется использовать базовые термины вместе с более сложными, допускается использование повторов и синонимов.
	% Не рекомендуется использование слишком сложных слов, слов в кавычках, слов с запятыми.
	% По возможности следует применять слова в основной форме именительного падежа единственного числа.
	% Рекомендуемое количество ключевых слов~-- 5-7, количество слов внутри ключевой фразы~-- не более 3.
	
\bigskip
\noindent
\textbf{Ключевые слова}: \emph {фМРТ; звуковой ряд; временной ряд; прогнозирование; причинно-следственный анализ;}
}

\titleEng
	[Prediction of neural activity with exogenous factors] % краткое название; не нужно, если полное название влезает в~колонтитул
    {Prediction of neural activity with exogenous factors}
\authorEng
	[F.\,S.~Author] % список авторов (не более трех) для колонтитула; не нужен, если основной список влезает в колонтитул
	{F.\,S.~Author, F.\,S.~Co-Author, and F.\,S.~Name} % основной список авторов, выводимый в оглавление
    [F.\,S.~Author$^1$, F.\,S.~Co-Author$^2$, and F.\,S.~Name$^{1, 2}$] % список авторов, выводимый в заголовок; не нужен, если он не отличается от основного
\thanksEng
    {The research was
     %partially
    	 supported by the Russian Foundation for Basic Research (grants 00-00-0000 and 00-00-00001).
    }
\organizationEng
    {$^1$Organization, address; $^2$Organization, address}
\abstractEng
     {
 %This is the template of the paper submitted to the journal ``Machine Learning and Data Analysis''.
		
	% \noindent
	% The title should be concise and informative. Titles are often used in information-retrieval systems. Avoid abbreviations and formulae where possible.
	% Please clearly indicate the last names and initials of each author and check that all names are accurately spelled. Present the authors' affiliation
	% addresses where the actual work was done.
	% Provide the full postal address of each affiliation, including the country name and, if available, the
	% e-mail address of each author.
	% Provide only institutional affiliation, department/division affiliation are not required.

	% \noindent
	% A concise and factual abstract is required.
	% The purpose of the abstract is to provide a summary~of the paper enabling the reader to decide whether or not to read the full text.
 %    	The abstract should state briefly the purpose of the research, the principal results and major conclusions.
 %    	An abstract is often presented separately from the article, so it must be able to stand alone.
 %    	For this reason, References should be avoided, but if essential, then cite the author(s) and year(s).
 %    	Also, non-standard or uncommon abbreviations should be avoided, but if essential they must be defined at their first mention in the abstract itself.
 %    	The requirements on the size of the abstract is about 200--300 words.
 %    	It should be provided in the next structured manner:
	
	% \noindent
	% \textbf{Background}:	One paragraph about the problem, existent approaches and its limitations.
	
	% \noindent
	% \textbf{Methods}: One paragraph about proposed method and its novelty.
	
	% \noindent
	% \textbf{Results}: One paragraph about major properties of the proposed method and experiment results if applicable.
	
	% \noindent
	% \textbf{Concluding Remarks}: One paragraph about the place of the proposed method among existent approaches.
		
	% \noindent
	% Immediately after the abstract, provide 5-7 keywords, avoiding general and plural terms and multiple concepts (avoid, for example, ``and'', ``of'').
	% Use keywords that are specific and that reflect what is essential about the paper.
	% Use keywords from the abstract, introduction and conclusion.
	% These keywords will be used for indexing purposes.
		
	\noindent
    	\textbf{Keywords}: \emph{keyword; keyword; more keywords, separated by ``;''}}

%данные поля заполняются редакцией журнала
% \doi{10.21469/22233792}
% \receivedRus{01.01.2017}
% \receivedEng{January 01, 2017}

\maketitle
\linenumbers

\section{Введение}

[the research goal (and its motivations),]
Работа посвящена восстановлению зависимости между снимками фМРТ
и звуковым рядом и улучшению понимания взаимосвязи между активностью мозга и внешними раздражителями. [TODO Новизна].

[the object of research (introduce main termini),]
Функциональная магнитно-резонансная томография (фМРТ) — это метод нейровизуализации, который измеряет активность мозга путем выявления изменений, связанных с кровотоком. Он дает представление о том, какие области мозга участвуют в конкретных психических процессах или задачах, измеряя изменения уровня оксигенации крови (сатурация). 

[the problem (what is the challenge)]
Существует несколько ограничений:  Временное и пространственное разрешение -- время измерений фМРТ происходит с задержкой, что затрудняет регистрацию быстрых нейронных событий. Шум -- сигналы фМРТ могут быть слабыми по сравнению с фоновым шумом, что может повлиять на точность результатов.


[methodology: literature review and state-of-the-art]
[the project tasks,]


[the proposed solution, its novelty, and advantages]
Одним из потенциальных решений является использование нейронных сетей, в частности, с использованием готовых архитектур как ResNet. Используя возможности глубокого обучения, мы сможем формализовать проблему и закодировать снимки фМРТ и звуковой ряд. 

[the profs and cons of recent works,]



[goal of the experiment, set up, data sets, workflow.]







\section{Название раздела}
Данный документ демонстрирует оформление статьи,
подаваемой в электронную систему подачи статей \url{http://jmlda.org/papers} для публикации в журнале <<Машинное обучение и анализ данных>>.
Более подробные инструкции по~стилевому файлу \texttt{jmlda.sty} и~использованию издательской системы \LaTeXe\
находятся в~документе \texttt{authors-guide.pdf}.
Работу над статьёй удобно начинать с~правки \TeX-файла данного документа.

Обращаем внимание, что данный документ должен быть сохранен в кодировке~\verb'UTF-8 without BOM'.
Для смены кодировки рекомендуется пользоваться текстовыми редакторами \verb'Sublime Text' или \verb'Notepad++'.

\paragraph{Название параграфа}
Разделы и~параграфы, за исключением списков литературы, нумеруются.

\section{Заключение}
Желательно, чтобы этот раздел был, причём он не~должен дословно повторять аннотацию.
Обычно здесь отмечают, каких результатов удалось добиться, какие проблемы остались открытыми.

%%%% если имеется doi цитируемого источника, необходимо его указать, см. пример в \bibitem{article}
%%%% DOI публикации, зарегистрированной в системе Crossref, можно получить по адресу http://www.crossref.org/guestquery/
\begin{thebibliography}{99}
% \bibitem{book}
%     \BibAuthor{Гуссенс~М., Миттельбах~Ф., Cамарин~А.}
%     \BibTitle{Путеводитель по пакету \LaTeX\ и~его расширению \LaTeXe} / Пер. с англ.~---
%     М.:~Мир, 1999. 606~с.
%     (\BibAuthor{Goossens M., Mittelbach F., Samarin A.}
%      \BibTitle{The \LaTeX\ companion}.~--- 2nd ed.~--- Reading, MA, USA: Addison-Wesley, 1994. 528 p.)

% \bibitem{article}
%     \BibAuthor{Загуренко~А.\,Г., Коротовских~В.\,А., Колесников~А.\,А., Тимонов~А.\,В., Кардымов~Д.\,В.}
%     Технико-экономическая оптимизация дизайна гидроразрыва пласта~//
%     \BibJournal{Нефтяное хозяйство}, 2008. Т.~11. \No\,1. С.~54--57.
% 	\BibDoi{10.3114/S187007708007}.

% \bibitem{webArticle}
% 	\BibAuthor{Blaga~P.\,A.}
% 	Commutative Diagrams with XY-pic II. Frames and Matrices~//
% 	\BibJournal{PracTEX J.}, 2007. Vol.\,4.
% 	URL: \BibUrl{https://tug.org/pracjourn/2007-1/blaga/blaga.pdf}.

% \bibitem{webResource}
% 	XYpic.
% 	URL: \BibUrl{http://akagi.ms.u-tokyo.ac.jp/input9.pdf}.
	
% \bibitem{inproceedingsRus}
% 	\BibAuthor{Усманов~Т.\,С., Гусманов~А.\,А., Муллагалин~И.\,З., Мухаметшина~Р.\,Ю., Червякова~А.\,Н., Свешников~А.\,В.}
% 	Особенности проектирования разработки месторождений с применением гидроразрыва пласта~//
% 	\BibJournal{Труды 6-го Междунар. симп. <<Новые ресурсосберегающие технологии недропользования и повышения нефтегазоотдачи>>}.~---
% 	М.:~Издательство, 2007. С.~267--272.

% \bibitem{inproceedingsEng}
%     \BibAuthor{Author~N.}
%     Paper title~//
%     \BibJournal{10th Conference (International) on Any Science Proceedings}.~---
%     Place of publication: Publisher, 2009. P.~111--122.

% \bibitem{techreport}
% 	\BibAuthor{Lambert~P.}
%   	\BibTitle{The title of the work}.
%   	Place of publication:~The institution that published, 1993.  Report~2.
 	
\end{thebibliography}

%%%% если имеется doi цитируемого источника, необходимо его указать, см. пример в \bibitem{article}
%%%% DOI публикации, зарегистрированной в системе Crossref, можно получить по адресу http://www.crossref.org/guestquery/.

\maketitleSecondary
\English
\begin{thebibliography}{99}
% \bibitem{book}
% 	\BibAuthor{Goossens,~M., F. Mittelbach, and A.~Samarin}. 1994.
% 	\BibTitle{The \LaTeX\ companion}.
% 	2nd ed.
% 	Reading, MA: Addison-Wesley. 528 p.

% \bibitem{article}
% 	\BibAuthor{Zagurenko,~A.\,G., V.\,A.~Korotovskikh, A.\,A.~Kolesnikov, A.\,V.~Timonov, and D.\,V.~Kardymon}. 2008.
% 	Tekhniko-ekonomicheskaya optimizatsiya dizayna gidrorazryva plasta
% 	[Technical and economic optimization of the design of hydraulic fracturing].
% 	\BibJournal{Neftyanoe Khozyaystvo} [Oil Industry] 11(1):54--57.
% 	\BibDoi{10.3114/S187007708007}. (In Russian)

% \bibitem{webArticle}
% 	\BibAuthor{Blaga,~P.\,A.} 2007.
% 	Commutative Diagrams with XY-pic II. Frames and Matrices.
% 	\BibJournal{PracTEX J.}  4.
% 	Available at: \BibUrl{https://tug.org/pracjourn/2007-1/blaga/blaga.pdf}
%     (accessed February 20, 2007).

% \bibitem{webResource}
% 	XYpic.
% 	Available at: \BibUrl{http://akagi.ms.u-tokyo.ac.jp/input9.pdf}
% 	(accessed April 09, 2015).

% \bibitem{inproceedingsRus}
% 	\BibAuthor{Usmanov,~T.\,S., A.\,A.~Gusmanov, I.\,Z.~Mullagalin, R.\,Yu.~Mukhametshina, A.\,N.~Chervyakova, and A.\,V.~Sveshnikov.} 2007.
% 	Osobennosti proektirovaniya razrabotki mestorozhdeniy s primeneniem gidrorazryva plasta
% 	[Features of the design of field development with the use of hydraulic fracturing].
% 	\BibJournal{6th Symposium (International) ``New Energy Saving Subsoil Technologies and the
% 	Increasing of the Oil and Gas Impact'' Proceedings}.
% 	Moscow:~Publisher. 267--272. (In Russian)
	   	
% \bibitem{inproceedingsEng}
%     \BibAuthor{Author,~N.} 2009.
%     Paper title.
%     \BibJournal{10th Conference (International) on Any Science Proceedings}.
%     Place of publication: Publisher. 111--122.
	
% \bibitem{techreport}
% 	\BibAuthor{Lambert,~P.} 1993.
%   	\BibTitle{The title of the work}.
%   	Place of publication:~The institution that published.  Report~2.
  	     	
\end{thebibliography}

\end{document}
